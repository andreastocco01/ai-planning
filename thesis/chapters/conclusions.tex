\chapter{Conclusions}
This thesis investigated heuristic approaches for solving delete-free planning problems, a simplified yet challenging fragment of classical planning.
By removing negative effects, delete-free planning yields a monotonic state space, which simplifies reasoning about reachability while preserving
much of the inherent difficulty of plan generation.

The study first revisited several well-known heuristics and implemented them within a common experimental framework. Building on these foundations,
a novel heuristic was proposed, designed to exploit structural properties of delete-free tasks more effectively. The performance of all algorithms
was systematically evaluated across a suite of benchmark instances, with a focus on both computational efficiency and the quality of the resulting plans.

The results highlight that the proposed heuristic achieves competitive and superior performance compared to established approaches,
confirming the potential of structural exploitation in delete-free domains. The analysis of randomization further revealed that the stability of solution
quality strongly depends on the informativeness of the heuristic: more informative heuristics consistently led to more stable outcomes across different
random seeds, while less informative ones exhibited greater variability. Finally, refinement strategies, whether based on re-solving subproblems with
Uniform Cost Search or on reapplying the heuristic itself to subproblems, did not yield substantial improvements. In both cases,
the refined plans largely overlapped in cost with the originals, indicating that such post-processing techniques are not particularly effective in this setting.

At the same time, the work opens several avenues for further investigation. A natural next step would be to explore local branching using an external
solver as a black box, which could allow richer refinements of partial solutions than the UCS-based approach adopted here.
Similarly, the pruning-enhanced variant of the max heuristic developed in this thesis could be integrated into A* search, enabling an optimal procedure
to benefit from early elimination of unhelpful actions. These directions point to practical ways of extending the contributions of this work,
while reinforcing the broader insight that delete-free planning offers a valuable testbed for designing and evaluating new heuristic strategies.
