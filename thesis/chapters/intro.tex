\chapter{Introduction}
\label{ch:intro}
\textit{Automated Planning} (or AI Planning) is a core subfield of Artificial Intelligence.
A planning problem, in general terms, can be defined as the task of finding a sequence
of actions that leads from a given initial state to a desired goal state.
While multiple formal models exist for defining planning problems, this thesis
focuses on the \textit{$SAS^+$ formalism} defined in Definition \ref{def:sas+}

\begin{definition}[$\text{SAS}^+$ Planning Task]
	\label{def:sas+}
	A $\text{SAS}^+$ planning task is a 5-tuple $\Pi = \langle V, s_0, s_*, A \rangle$ with
	the following components:
	\begin{itemize}
		\item \(V\): finite set of state variables \(v\),
		      each with finite domain \textit{dom(v)}
		\item \(s_0\): variable assignment defining the initial state
		\item \(s_*\): partial variable assignment defining the goal
		\item \(A\): finite set of actions (or operators),
		      where each action $a \in A$ has the following components:
		      \begin{itemize}
			      \item Preconditions \textit{pre(a)}: partial variable assignment
			      \item Effects \textit{eff(a)}: partial variable assignment
			      \item Cost \textit{cost(a)}: non-negative real number
		      \end{itemize}
	\end{itemize}
\end{definition}

For an action $a \in A$ to be applicable, all of its preconditions
\textit{pre(a)} must be satisfied in the current state. Once applied, its effects
\textit{eff(a)} are used to update the current state accordingly.
In the simplified class of problems considered in this thesis—referred to as \textit{Delete-Free}—actions do not have any negative effects.
This means that once a fact becomes true in a state, it remains true in all subsequent states.
The solutions to planning problems are called \textit{plans}, and they are formally defined in Definition \ref{def:plan}

\begin{definition}[Plan]
	\label{def:plan}
	A plan for a planning problem is a sequence of actions occurring as labels on a path
	from the initial state to a goal state.
	The cost of a plan $\langle a_1, a_2, \dots, a_n \rangle$ is $\sum_{i = 1}^n cost(a_i)$.
	A plan is optimal if it has minimal cost.
\end{definition}

There exist optimal planners, such as \textit{Fast-Downward}, that are guaranteed to find the optimal solution to a given problem.
In contrast, this thesis focuses on \textit{heuristic} algorithms which—by design— do not
guarantee optimality, but aim to efficiently find near-optimal solutions in significantly less time.
