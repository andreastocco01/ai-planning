\chapter{Results}
\label{ch:results}
This chapter presents an evaluation of the algorithms introduced in Chapter \ref{ch:heuristics}.
To assess their performance, a cumulative distribution plot was used.
This type of plot shows the percentage of instances for which an algorithm found a plan with a
primal gap below the threshold indicated on the x-axis.
The best algorithm is the one whose curve lies above all the others.
The plots presented in the following sections will help illustrate this concept.
As described in Chapter \ref{ch:methodology}, each algorithm was run on every instance in the
testbed 10 times, each with a different random seed.
There are 3,104 instances, so each algorithm was executed a total of 31,040 times.
To limit execution time, each run was given a time limit of 60 seconds. The instances
come from different domains; some are very large in terms of number of actions, while others are
relatively small.
Additionally, since some algorithms are more efficient than others, not all the instances
were solved by every algorithm. Table \ref{tab:timelimit} shows, for each algorithm,
the number of instances it was able to solve within the time limit.

\begin{table}[h!]
	\centering
	\begin{tabular}{|l|r|}
		\hline
		\textbf{Heuristic}        & \textbf{Not found per timelimit} \\
		\hline
		Random                    & 1123                             \\
		Greedy                    & 1076                             \\
		Greedy + Pruning          & 1714                             \\
		Max Heuristic + Lookahead & 9199                             \\
		Shortest Path             & 851                              \\
		\hline
	\end{tabular}
	\caption{Number of instances not solved within the time limit (out of 31,040 total instances).}
	\label{tab:timelimit}
\end{table}

By simply examining this table, we can identify the most effective algorithms.
The Lookahead process is computationally expensive, resulting in approximately one-third of the instances not being solved.
In contrast, the efficient implementation of the Shortest Path heuristic appears to be the most capable,
solving the highest number of instances.

\section{Random vs Greedy vs Greedy + Pruning vs Shortest Path}
This section presents an evaluation of four planning strategies, ranging from uninformed to increasingly informed approaches:
Random, Greedy, Greedy + Pruning, and Shortest Path.
Random and Greedy exhibit similar efficiency, as both are able to solve a large number of instances. Greedy + Pruning is
slightly slower but employs a strategy with the potential to yield higher-quality solutions. Ultimately, the Shortest Path approach
solves the highest number of instances. The next step is to evaluate the quality of the solutions produced by each strategy.
