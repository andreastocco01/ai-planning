\chapter{Results}
\label{ch:results}
This chapter presents an evaluation of the algorithms introduced in Chapter \ref{ch:heuristics}.
To assess their performance, a cumulative distribution plot was used.
This type of plot shows the percentage of instances for which an algorithm found a plan with a
primal gap below the threshold indicated on the x-axis.
The best algorithm is the one whose curve lies above all the others.
The plots presented in the following sections will help illustrate this concept.
As described in Chapter \ref{ch:methodology}, each algorithm was run on every instance in the
testbed 10 times, each with a different random seed.
There are 3,104 instances, so each algorithm was executed a total of 31,040 times.
To limit execution time, each run was given a time limit of 60 seconds. The instances
come from different domains; some are very large in terms of number of actions, while others are
relatively small.
Additionally, since some algorithms are more efficient than others, not all the instances
were solved by every algorithm. Table \ref{tab:timelimit} shows, for each algorithm,
the number of instances it was able to solve within the time limit.

\begin{table}[h!]
	\centering
	\begin{tabular}{|l|r|}
		\hline
		\textbf{Heuristic}        & \textbf{Not found per timelimit} \\
		\hline
		Random                    & 1123                             \\
		Greedy                    & 1076                             \\
		Greedy + Pruning          & 1714                             \\
		Max Heuristic + Lookahead & 9199                             \\
		Shortest Path             & 851                              \\
		\hline
	\end{tabular}
	\caption{Number of instances not solved within the time limit (out of 31,040 total instances).}
	\label{tab:timelimit}
\end{table}

By simply examining this table, we can identify the most effective algorithms.
The lookahead process is computationally expensive, resulting in approximately one-third of the instances not being solved.
In contrast, the efficient implementation of the shortest path heuristic appears to be the most capable,
solving the highest number of instances.

\section{Random vs Greedy vs Greedy + Pruning vs Shortest Path}
This section presents an evaluation of four planning strategies, ranging from uninformed to increasingly informed approaches:
random, greedy, greedy + pruning, and shortest path.
Random and greedy exhibit similar efficiency, as both are able to solve a large number of instances. Greedy + pruning is
slightly slower but employs a strategy with the potential to yield higher-quality solutions. Ultimately, the shortest path approach
solves the highest number of instances. The next step is to evaluate the quality of the solutions produced by each strategy.
Figure \ref{fig:rgps} presents the cumulative distribution plot for the evaluated algorithms.

\begin{figure}[h!]
	\centering
	\includegraphics[width=\textwidth]{images/algs0124.png}
	\caption{Cumulative distribution plot for Random, Greedy, Greedy + Pruning, and Shortest Path strategies.}
	\label{fig:rgps}
\end{figure}

For instance, to better interpret the plot, consider the greedy strategy: the point on its curve corresponding to
a primal gap threshold of 60\% is close to 20\%. This indicates that it achieves a solution with a primal gap less than
or equal to 60\% in approximately 20\% of the problem instances.
Conversely, the shortest path strategy found the best known solution (i.e., a primal gap of 0\%) in approximately
25\% of the instances.
Now that the plot has been properly explained, it is possible to identify the most effective heuristic.
The curve that remains above all the others represents the best-performing strategy, as it indicates that the corresponding
algorithm consistently found solutions within the primal gap thresholds for a larger number of instances.
As expected, random and greedy stategies typically do not produce high-quality plans. Pruning the useless actions appears to be
an effective strategy, as it allows the planner to discard actions that are known to be unhelpful whenever multiple
options with the same cost are available.
Shortest Path builds on pruning by also estimating the distance to the goal for each action, resulting in a more
informed and effective selection of actions to include in the plan.

\section{Greedy + Pruning vs Hmax + Lookahead}
As shown in the previous section, pruning proves to be an effective strategy.
The initial idea was to combine pruning with the max heuristic. However, due to the nature of the heuristic’s definition,
this approach effectively reduces to a greedy + pruning strategy. To better evaluate its potential,
a lookahead mechanism was implemented to assess whether combining pruning with the max heuristic could lead
to a more promising solution. This section presents an evaluation of the performance of these two algorithms.
The lookahead process is computationally expensive; as a result, the algorithm fails to find a plan for approximately
one-third of the instances. Figure \ref{fig:algs23} presents the cumulative distribution plot comparing the performance
of these two algorithms.

\begin{figure}[h!]
	\centering
	\includegraphics[width=\textwidth]{images/algs23.png}
	\caption{Cumulative distribution plot for Greedy + Pruning, and Max Heuristic + Lookahead strategies.}
	\label{fig:algs23}
\end{figure}

Unfortunately, this plot alone cannot definitively answer which of the two algorithms produces better-quality plans.
The high number of unsolved instances by the hmax + lookahead algorithm can skew the results: greedy + pruning may
appear superior simply because it solves more instances, not necessarily because it generates better plans.
There is no guarantee that the plans found by greedy + pruning are of higher quality than those returned by hmax + lookahead.
Figure \ref{fig:algs234_solved_all} shows a comparison between the algorithms, considering only the instances that were successfully
solved by all of them.

\begin{figure}[h!]
	\centering
	\includegraphics[width=\textwidth]{images/algs234_solved_all.png}
	\caption{Cumulative distribution plot for Greedy + Pruning, Max Heuristic + Lookaheado, and Shortest Path strategies,
		considering only the instances solved by all of them.}
	\label{fig:algs234_solved_all}
\end{figure}

The curve for shortest path is included as a reference. Focusing only on the instances solved by all the presented algorithms
is particularly valuable: in this case, the curves are not skewed by instances that some strategies failed to solve.
This allows us to assess only the quality of the generated plans. From this comparison, it is evident that shortest path
is the best-performing algorithm overall—both in terms of solution quality and the number of problems it can solve.
Additionally, the performance of hmax + lookahead, which surpasses that of greedy + pruning, confirms that pruning combined
with max heuristic can be an effective strategy when paired with a lookahead mechanism.
Although Lookahead remains a slow algorithm, these results suggest that the revised version of the max heuristic|enhanced
with a pruning mechanism|has the potential to produce high-quality solutions if used within a more powerful search strategy
like \verb|A*|.

\section{Different Versions of Backward Propagation}
As explained in Section \ref{sec:shortestpath}, the core idea behind the algorithm is to back-propagate the estimated cost
of reaching the goal state from its constituent facts to those in the current state. Previously, an algorithm was introduced
that propagates the minimum cost among an action's effects—effectively computing the shortest path to the goal.
With minimal modification, alternative strategies can be explored, such as propagating the \textit{maximum effect cost} or the
\textit{sum of all effect costs} to preceding actions. These variations are evaluated in Figure \ref{fig:backprop} to determine which
propagation strategy yields better results.

\begin{figure}[h!]
	\centering
	\includegraphics[width=\textwidth]{images/algs456.png}
	\caption{Cumulative distribution plot for different backpropagation strategies}
	\label{fig:backprop}
\end{figure}

Back-propagating the maximum cost among an action’s effects can be interpreted as assigning the action a cost based on
the worst-case path to the goal state. Conversely, back-propagating the sum of the effects' costs reflects a perspective
where all possible outcomes of the action contribute to estimating the distance to the goal. Each approach offers a different
trade-off between pessimism and comprehensiveness in cost estimation.
The Shortest Path heuristic continues to deliver the best performance overall. In contrast, both the max-cost and sum-cost
backpropagation strategies exhibit similar behavior, with only marginal differences between them.

\section{Randomization Analysis}
All the presented algorithms share a common trait: when multiple actions have the same minimum heuristic cost,
the next action is selected at random among them.
This section analyzes the impact of random tie-breaking on the quality of the returned plans. As mentioned earlier,
each algorithm was executed on every instance using 10 different random seeds. Varying the seed affects the algorithm’s
behavior only in cases where multiple actions share the same heuristic cost, influencing which action is selected among them.
Figure \ref{fig:rand_rand} presents the analysis of the impact of randomization on the random strategy.

\begin{figure}[h!]
	\centering
	\includegraphics[width=\textwidth]{images/randomization_random.png}
	\caption{Effect of random seed on the outcome variability of the Random approach.}
	\label{fig:rand_rand}
\end{figure}

It is a \textit{logarithmic plot}: it uses a logarithmic scale on both axis to display data with a wide range of value.
Each point represents one instance in the testbed, and its coordinates correspond to the minimum and maximum plan costs found
by the algorithm across different random seeds.
