\chapter{Capitolo di prova}

Questo capitolo può servire come riferimento per alcuni comandi utili di \LaTeX.

\section{Esempio di sezione}

\subsection{Esempio di sottosezione}

\begin{displayquote}[J. R. R. Tolkien]
	The Lord of the Rings \textit{is one of these things: \\ if you like it, you do; \\ if you don't, then you boo!}
\end{displayquote}

Questa qui sopra è una citazione. Bella, vero?

Questo è un riferimento: \cite{lamport1986latex}.

Questa è un'espressione matematica in riga: $y = f(x)$. Questa è un'espressione matematica in un blocco a sé:

\begin{equation}
	\int\limits_{-\infty}^\infty e^{-x^2}dx = \sqrt{\pi}
\end{equation}

Questo è codice in riga: \verb|for(int i = 0; i < 1; i++);|. Questo è un listato semplice, estratto da \textit{The C programming language} (1988), p. 6:

\begin{verbatim}
    #include <stdio.h>

    main( )
    {
        printf("hello, world\n");
    }
\end{verbatim}

Questo è un listato sbrilluccicoso:

\begin{lstlisting}[mathescape,gobble=2]{python}
    # can your editor do this? $\displaystyle{f(x) = m + \sum\limits_{k=1}^\infty A_k \cos(k x - \varphi_k)}$

    import this

    print('hello, world')
\end{lstlisting}
